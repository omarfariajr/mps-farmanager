\chapter[Plano de Monitoramento a ser executado pela Gerência de Projeto]{Plano de Monitoramento a ser executado pela Gerência de Projeto}

Mensalmente o Gerente do Projeto desenvolverá um relatório formal de acompanhamento do projeto e da melhoria estipulada, que será disponibilizada para todos.

O relatório confeccionado será uma espécie de análise, relatando principalmente erros e acertos, e problemas que ocorreram. Em caso de detecção de abordagens erradas, ou de pontos críticos, ações podem ser tomadas, como: realocação de atividades, conversas com os outros papéis e replanejamento. Assim, fica caracterizado o PDCA nas atividades do Gerente do Projeto: após o planejamento e do desenvolvimento (Plan e Do), ele analisa os resultados (Check) através do relatório, e toma atitudes quanto aos resultados coletados (Act), sendo aplicado então como uma melhoria contínua.