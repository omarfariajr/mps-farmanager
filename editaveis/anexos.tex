\begin{anexosenv}

\partanexos

\chapter{Documento de Visão}

\subsection{Introdução}

\subsubsection{Finalidade}

A finalidade deste documento é fornecer uma visão geral da aplicação de aquisição e manipulação dos dados do motor a combustão, apresentando uma visão das macro-funcionalidades do software. Além disso, objetiva-se apresentar as razões pelas quais o sistema será construído.

\subsubsection{Escopo}

A aplicação destina-se ao suporte ao usuário da bancada, com intuito de fornecer a visualização de informações a cerca do funcionamento do motor no momento da análise de forma gráfica. Tais informações são: Temperatura do óleo do motor; Temperatura do ar no coletor de admissão; Pressão do ar no coletor de admissão; Informações de emissão e mistura sonda/lâmbda.

\subsection{Posicionamento}

\subsubsection{Descrição do problema}

\begin{table}[h!]
	\centering
	\caption{Descrição do problema.}
	\label{descricaoproblema}
	\begin{tabular}{|l|l|}
		\hline
		\textbf{O problema de}                                                    & \begin{tabular}[c]{@{}l@{}}Dificuldade de se analisar os parâmetros e características de um,\\ motor em funcionamento.\end{tabular}                                      \\ \hline
		\textbf{Afeta}                                                            & Estudantes e professores do curso de engenharia automotiva.                                                                                                              \\ \hline
		\textbf{Cujo impacto é}                                                   & \begin{tabular}[c]{@{}l@{}}Impossibilidade, por parte dos alunos, de conhecer e analisar\\ parâmetros e características de um motor a combustão na prática.\end{tabular} \\ \hline
		\textbf{\begin{tabular}[c]{@{}l@{}}Uma boa solução \\ seria\end{tabular}} & \begin{tabular}[c]{@{}l@{}}Utilizar recursos gráficos, por meio de um software, para dar \\ suporte na análise dos dados de um motor em funcionamento.\end{tabular}      \\ \hline
	\end{tabular}
\end{table}

\newpage

\subsubsection{Sentença de posição do produto}

\begin{table}[h!]
	\centering
	\caption{Sentença de posição do produto}
	\label{my-label}
	\begin{tabular}{|l|l|}
		\hline
		\textbf{Para}            & Compor a bancada de análise                                                                                                                                                              \\ \hline
		\textbf{Que}             & \begin{tabular}[c]{@{}l@{}}Necessita de um sistema de software para dar suporte na visualização\\ das informações do motor.\end{tabular}                                                 \\ \hline
		\textbf{O}               & Software de Aquisição e Processamento de Dados de Motor                                                                                                                                  \\ \hline
		\textbf{Que}             & Auxiliará o usuário da bancada a visualizar as informações do motor                                                                                                                      \\ \hline
		\textbf{Ao contrário de} & Realizar as análises com o veículo completo                                                                                                                                              \\ \hline
		\textbf{A aplicação}     & \begin{tabular}[c]{@{}l@{}}Promoverá uma interface gráfica entre o usuário da bancada e o motor \\ ao qual apresentará as informações referentes ao motor em funcionamento.\end{tabular} \\ \hline
	\end{tabular}
\end{table}


\subsection{Decrição dos Envolvidos e dos Usuários}

\subsubsection{Resumo dos envolvidos}

Informações dos stakeholders do projeto de desenvolvimento do software.

\begin{table}[h!]
	\centering
	\caption{Descrição dos envolvidos e dos usuários.}
	\label{descricaoDosEnvolvidos}
	\begin{tabular}{|l|l|l|}
		\hline
		\textbf{Nome}                                                                     & \textbf{Descrição}                                                                                                                                                                              & \textbf{Responsabilidades}                                                                                                                               \\ \hline
		\begin{tabular}[c]{@{}l@{}}Desenvolvedores do \\ software de análise\end{tabular} & \begin{tabular}[c]{@{}l@{}}Alunos de graduação do\\  curso de Engenharia de\\  Software\end{tabular}                                                                                            & \begin{tabular}[c]{@{}l@{}}Análise dos requisitos do \\ software, modelar \\ arquitetura do software,\\ implementar e implantar\\ software.\end{tabular} \\ \hline
		Professores                                                                       & \begin{tabular}[c]{@{}l@{}}Professores da disciplina de \\ Projeto Integrador 2\end{tabular}                                                                                                    & \begin{tabular}[c]{@{}l@{}}Acompanhar e avaliar o \\ desenvolvimento do projeto\end{tabular}                                                             \\ \hline
		\begin{tabular}[c]{@{}l@{}}Equipe de desenvolvimento \\ da bancada\end{tabular}   & \begin{tabular}[c]{@{}l@{}}Alunos de graduação do\\ curso de Engenharia de \\ Energia, Eletrônica, \\ Automotiva, Aeroespacial \\ que compõem a equipe\\ desenvolvedora da bancada\end{tabular} & \begin{tabular}[c]{@{}l@{}}Atuar como clientes,\\ apresentando os problemas\\ que tem que ser resolvidos\\  através da aplicação.\end{tabular}           \\ \hline
	\end{tabular}
\end{table}

\newpage

\subsubsection{Resumo dos usuários}

Informações dos usuários finais do software.

\begin{table}[h!]
	\centering
	\caption{Usuários finais}
	\label{usuariosfinais}
	\begin{tabular}{|l|l|l|}
		\hline
		\textbf{Nome} & \textbf{Descrição}                                                                       & \textbf{Responsabilidades}                                                                                                                \\ \hline
		Professores   & \begin{tabular}[c]{@{}l@{}}Professores do curso de \\ Engenharia Automotiva\end{tabular} & \begin{tabular}[c]{@{}l@{}}Realizar o acompanhamento \\ geral das informações \\ apresentadas do motor e\\ gerar relatórios.\end{tabular} \\ \hline
		Alunos        & \begin{tabular}[c]{@{}l@{}}Alunos do curso de \\ Engenharia Automotiva\end{tabular}      & \begin{tabular}[c]{@{}l@{}}Realizar o acompanhamento\\ geral das informações \\ apresentadas do motor \\ e gerar relatórios.\end{tabular} \\ \hline
	\end{tabular}
\end{table}

\subsubsection{Ambiente do usuário}

O software será utilizado diretamente na bancada, pois trata-se de uma aplicação desktop para o raspbian, sistema operacional baseado em debian que servirá como plataforma para o software. 

\subsubsection{Principais necessidades dos usuários ou dos envolvidos}

\begin{table}[h!]
	\centering
	\caption{Necessidades dos usuários ou dos envolvidos}
	\label{necessidadesDosUsuariosOuDosEnvolvidos}
	\begin{tabular}{|l|l|l|}
		\hline
		\textbf{Necessidade}                                                                                                                        & \textbf{Solução atual} & \textbf{Solução proposta}                                                                                                                                      \\ \hline
		\begin{tabular}[c]{@{}l@{}}Realizar o \\ acompanhamento geral\\ das informações \\ apresentadas do motor e\\ gerar relatórios.\end{tabular} & Não há.                & \begin{tabular}[c]{@{}l@{}}Exposição dos dados\\ coletados pelos sensores\\ acoplados ao motor, \\ geração de gráficos, \\ geração de relatórios.\end{tabular} \\ \hline
	\end{tabular}
\end{table}

\subsection{Visão Geral do Produto}

\subsubsection{Perspectiva do produto}

Espera-se que o software sirva de auxílio à bancada de testes de um motor a combustão, permitindo uma melhor e mais agradável visualização do funcionamento do equipamento, resultando numa maior facilidade no aprendizado a partir dos alunos.

\subsubsection{Resumo dos recursos}

Os recursos que o sistema deve conter são funcionalidades importantes para resolver o problema e as necessidades do cliente. Este software conta com as seguintes funcionalidades:

\begin{itemize}
	\item Coleta e Armazenamento de dados - Esta funcionalidade permite que o usuário salve os dados coletados na análise.
	\item Visualização dos resultados - Esta funcionalidade permite que o usuário visualize os resultados da análise realizada por meio de gráficos intuitivos.
\end{itemize}

\subsection{Recursos do Produto}

\begin{itemize}
	\item O sistema deve coletar os dados do motor em “tempo real” transmitidos pela MSP430;
	\item O sistema deve realizar o tratamento dos dados para apresentá-los ao usuário;
	\item O sistema deve plotar gráficos a partir dos dados captados;
	\item O sistema deve possuir um botão para que o usuário tenha a opção de salvar todos os dados da análise em um banco de dados;
	\item O sistema deve possuir a opção de gerar um relatório.
\end{itemize}

\subsection{Restrições}

Nesta seção serão apresentados as restrições de design, restrições externas, como requisitos operacionais ou regulamentares e outras dependências.

São restrições do produto:

\begin{itemize}
	\item O software deverá ser embarcado em uma RaspBerry PI 3 B;
	\item O software deverá possuir rotinas para a comunicação serial com um microcontrolador MSP430;
	\item O software deverá ser, em essência, uma aplicação Desktop;
	\item O software não deverá realizar o controle de partida e aceleração do motor por questões de segurança.
\end{itemize}

\chapter{Segundo Anexo}

Texto do segundo anexo.

\end{anexosenv}

