\chapter[Plano de Monitoramento a ser executado pela Gerência Sênior]{Plano de Monitoramento a ser executado pela Gerência Sênior}

O Gerente de Processos de Software e o Grupo de Trabalho Técnico gerarão documentos formais mensalmente, objetificando o andamento do projeto como um todo, para que o Gerente Sênior possa tomar decisões precisas para os possíveis problemas apresentados. A depender dos resultados dos documentos apresentados, a frequência do processo de geração e apresentação de documentos por parte dos outros papéis ao Gerente Sênior pode ser aumentada, passando a ser quinzenalmente.

Além disso, utilizando as ferramentas de comunicação adotadas pelo time, reuniões virtuais ou presenciais podem ser convocadas pelo Gerente Sênior, em caso de pendências pontuais a serem resolvidas ou esclarecimentos relâmpagos.

Critérios e metas serão definidos, e, a partir de uma análise, atitudes serão tomadas quando o cumprimento destes não parecer tangível. Entre essas atitudes, ressaltam-se: reuniões, mudanças no escopo, realocação de RH nas atividades, e reunião com especialistas.